\documentclass[11pt]{article}
\usepackage{graphicx}
\usepackage{float}
\usepackage{amsmath}
\usepackage{amsfonts}
\usepackage[brazilian]{babel}
\usepackage[utf8]{inputenc}
\usepackage[backend=biber]{biblatex}
\usepackage{csquotes}
%\usepackage{docmute}
\usepackage{array}
\usepackage{multicol}
\usepackage{geometry}
\usepackage[T1]{fontenc}

\addbibresource{proposta.bib}

\newcommand{\fromeng}[1]{\footnote{do inglês: \textit{#1}}}
\newcommand{\tit}[1]{\textit{#1}}
\newcommand{\tbf}[1]{\textbf{#1}}
\newcommand{\ttt}[1]{\texttt{#1}}

\newcolumntype{C}[1]{>{\centering\let\newline\\\arraybackslash\hspace{0pt}}m{#1}}

\begin{document}

\begin{titlepage}
	\centering
	{\scshape\Large Plano de Trabalho\par}
	\vspace{1.5cm}
	{\huge\bfseries Um contador de \tit{hashtags} em tempo real para 
		\tit{twitter} usando \tit{Hadoop} e \tit{Spark}\par}
	\vspace{1cm}
	{\itshape Erik de Godoy Perillo - RA135582 - Grupo 17\par}
	\vspace{0.5cm}
	\begin{abstract}
		Neste documento, descreve-se a proposta para uma aplicação das 
		ferramentas vistas na disciplina 
		MC855 - Projeto em Sistemas de Computação.
	\end{abstract}
	\vfill
	Universidade Estadual de Campinas 
	\vfill
	{\large \today\par}
\end{titlepage}

\newgeometry{margin=1in}
\begin{multicols}{2}

\section{Introdução}
Redes sociais são uma fonte quase interminável de dados para \tit{Big Data}.
O colossal número de pessoas a usar esse sistema atualmente, juntamente com
técnicas avançadas de análise de grande volume de dados permite a extração
de uma quantidade enorme de informação útil.
Neste trabalho, a ideia é fazer um contador de \tit{hashtags} no \tit{twitter}
em tempo real, com possibilidade de filtragem por local/palavras.

\section{Frameworks}
A aplicação será possível tendo-se como base três componentes principais:

\begin{itemize}
	\item \tbf{Twitter Streaming API}~\cite{twit-stream} - 
		A API do \tit{twitter} é usada para a obtenção de
		mensagens da rede social em tempo real. Com a biblioteca, é possível
		obter uma amostra estatisticamente significativa das mensagens
		públicas do \tit{website}, com a opção de filtragem de mensagens por
		diversos critérios, como localização, idioma, palavras-chave etc.

	\item \tbf{Apache Hadoop}~\cite{hadoop} -
		\tit{Framework} amplamente usado hoje em dia com um ecossistema para
		\tit{Big Data} que possibilita a concepção de sistemas distribuídos
		com alto desempenho e confiabilidade.

	\item \tbf{Apache Spark}~\cite{spark} -
		Módulo que fica logicamente -- em geral -- acima da camada do 
		\tit{Hadoop}, extendendo-o com novas funcionalidades. Uma delas é
		a habilidade de lidar com \tit{streaming} de dados que chegam de modo
		contínuo em tempo real.
\end{itemize}

\section{Comportamento da aplicação}
A aplicação irá receber um \tit{stream} contínuo de dados do \tit{twitter}
através da API \tit{Twitter Streaming}. Será dada atenção especial ao corpo
das mensagens que chegam no fluxo. 

As mensagens serão enviadas então para o \tit{Spark}, que irá processar os dados
e analizar as \tit{hashtags} das mensagens. 
No fim, será possível investigar em tempo real quais as \tit{hashtags} mais
utilizadas atualmente advindas de mensagens com possíveis critérios de filtro,
como local, idioma, palavras-chave etc.

\printbibliography

\end{multicols}

\end{document}
